\documentclass{article}[14pt]
\usepackage[T2A]{fontenc}
\usepackage{geometry}
\usepackage{fancyhdr}
\usepackage{tocloft}
\usepackage{hyperref}
\usepackage{titlesec}
\usepackage{graphicx}
\usepackage{caption}
\usepackage{wrapfig}
\usepackage[english, russian]{babel}
\geometry{
	a4paper,
	top=20mm, 
	right=25mm, 
	bottom=25mm, 
	left=25mm
}
\usepackage{amssymb}
\usepackage{fancyvrb}
\usepackage{fvextra}
\usepackage{ dsfont }
\usepackage{amsmath}
\usepackage{listings}
\usepackage{xcolor}
\usepackage{listingsutf8}
\usepackage{pdfpages}
\usepackage{lipsum}
\usepackage{makecell}
\usepackage{float}
\usepackage{longtable}


\lstset{ 
	basicstyle=\ttfamily\small,
	keywordstyle=\color{blue},
	inputencoding=utf8,              
	extendedchars=true,
	stringstyle=\color{red}
	commentstyle=\color{green},
	showspaces=false,         
	showstringspaces=false,  
	numbers=left,              
	numberstyle=\tiny\color{gray},
	breaklines=true,          
	frame=single,           
	captionpos=b,             
	tabsize=4,
	escapeinside={(*@}{@*)}, % для работы с нестандартными текстами
	literate={ё}{{\"o}}1 {Ё}{{\"O}}1 {«}{{\guillemotleft}}1 {»}{{\guillemotright}}1
	{А}{{\CYRA}}1 {Б}{{\CYRB}}1 {В}{{\CYRV}}1 {Г}{{\CYRG}}1 {Д}{{\CYRD}}1 
	{Е}{{\CYRE}}1 {Ж}{{\CYRZH}}1 {З}{{\CYRZ}}1 {И}{{\CYRI}}1 {Й}{{\CYRISHRT}}1 
	{К}{{\CYRK}}1 {Л}{{\CYRL}}1 {М}{{\CYRM}}1 {Н}{{\CYRN}}1 {О}{{\CYRO}}1 
	{П}{{\CYRP}}1 {Р}{{\CYRR}}1 {С}{{\CYRS}}1 {Т}{{\CYRT}}1 {У}{{\CYRU}}1 
	{Ф}{{\CYRF}}1 {Х}{{\CYRH}}1 {Ц}{{\CYRC}}1 {Ч}{{\CYRCH}}1 {Ш}{{\CYRSH}}1 
	{Щ}{{\CYRSHCH}}1 {Ъ}{{\CYRHRDSN}}1 {Ы}{{\CYRERY}}1 {Ь}{{\CYRSFTSN}}1 
	{Э}{{\CYREREV}}1 {Ю}{{\CYRYU}}1 {Я}{{\CYRYA}}1 
	{а}{{\cyra}}1 {б}{{\cyrb}}1 {в}{{\cyrv}}1 {г}{{\cyrg}}1 {д}{{\cyrd}}1 
	{е}{{\cyre}}1 {ж}{{\cyrzh}}1 {з}{{\cyrz}}1 {и}{{\cyri}}1 {й}{{\cyrishrt}}1 
	{к}{{\cyrk}}1 {л}{{\cyrl}}1 {м}{{\cyrm}}1 {н}{{\cyrn}}1 {о}{{\cyro}}1 
	{п}{{\cyrp}}1 {р}{{\cyrr}}1 {с}{{\cyrs}}1 {т}{{\cyrt}}1 {у}{{\cyru}}1 
	{ф}{{\cyrf}}1 {х}{{\cyrh}}1 {ц}{{\cyrc}}1 {ч}{{\cyrch}}1 {ш}{{\cyrsh}}1 
	{щ}{{\cyrshch}}1 {ъ}{{\cyrhrdsn}}1 {ы}{{\cyrery}}1 {ь}{{\cyrsftsn}}1 
	{э}{{\cyrerev}}1 {ю}{{\cyryu}}1 {я}{{\cyrya}}1               
}
\begin{document}
	
	\begin{flushright}
		{\Large{Шишмаков Владимир Олегович, гр. №5130201/30101}}
	\end{flushright}

    \begin{center}
        {\LARGE{\textbf{Программный модуль 1}}}
    \end{center}

    \large

    \section{Описание}

    \noindent \textbf{Название:} Реализация длинной арифметики для целых чисел.

    \noindent \textbf{Дано:}

    \begin{itemize}
        \item $M$ --- целое число разрядов.
        \item $N$ --- целое число, характеризующее систему счисления.
        \item $a, b$ --- числа, над которыми производятся операции.
    \end{itemize}

    \noindent \textbf{Требуется:} Реализовать операции сложения, вычитания, умножения и целочисленного деления над целыми числами.

    \noindent \textbf{Ограничения:}

    \begin{itemize}
        \item $N > 0$
        \item $M > 0$
        \item $a \geq 0$
        \item $b \geq 0$
    \end{itemize}
    
    \noindent \textbf{Спецификация:}

    {\large{
    \begin{longtable}{|p{0.3cm}|p{3cm}|p{5cm}|p{6cm}|}
    \hline
    \textbf{№} & \textbf{Вход} & \textbf{Выход} & \textbf{Реакция программы} \\ \hline
    1 & $a = 15000000000; b = 1000000000; M = 10; N = 2$ & a = [0, 0] --- 0;
b = [0, 0] --- 0;
Sum: [0, 0] --- 0;
Sub: [0, 0] --- 0;
Mul: [0, 0] --- 0;
Div: [0] --- 0; & Программа принимает на вход числа a и b приводит их в формат длинной 
арифметики и выполняет над ними 4 операции. Выводит числа a и b, а также результаты операций над ними \\ \hline
    2 & $a = 2441; b = 2111; M = 10; N = 2$ & a = [4, 1] --- 41;
b = [1, 1] --- 11;
Sum: [5, 2] --- 52;
Sub: [3, 0] --- 30;
Mul: [5, 1] --- 51;
Div: [0, 3] --- 3; & Программа принимает на вход числа a и b приводит их в формат длинной 
арифметики и выполняет над ними 4 операции. Выводит числа a и b, а также результаты операций над ними \\ \hline
    3 & $a = 2441; b = 0; M = 10; N = 2$ & a = [4, 1] --- 41;
b = [0, 0] --- 0;
Sum: [4, 1] --- 41;
Sub: [4, 1] --- 41;
Mul: [0, 0] --- 0;
Div: [0] --- 0; & Программа принимает на вход числа a и b приводит их в формат длинной 
арифметики и выполняет над ними 4 операции. Выводит числа a и b, а также результаты операций над ними \\ \hline
    \end{longtable}}}

    \pagebreak

    \section{Исходный код}

    \begin{lstlisting}
M = 10
N = 2


def normalize(num, M=None, N=None):
    while len(num) > 1 and num[0] == 0:
        num = num[1:]
    return num


def to_int(num, M):
    num = normalize(num)
    result = 0
    for digit in num:
        result = result * M + digit
    return result


def from_int(num, M, N):

    mod = M ** N
    num %= mod

    digits = []
    for _ in range(N):
        digits.append(num % M)
        num //= M
    digits.reverse()
    return digits  


def compare(first_value, second_value):
    a = normalize(first_value[:])
    b = normalize(second_value[:])
    if len(a) > len(b):
        return 1
    if len(a) < len(b):
        return -1
    for x, y in zip(a, b):
        if x > y:
            return 1
        if x < y:
            return -1
    return 0


def sum(first_value, second_value, M, N):
    a = first_value[-N:]
    b = second_value[-N:]
    max_len = max(len(a), len(b))
    a = [0] * (max_len - len(a)) + a
    b = [0] * (max_len - len(b)) + b

    carry = 0
    res = [0] * (max_len + 1)
    for i in range(max_len - 1, -1, -1):
        s = a[i] + b[i] + carry
        res[i + 1] = s % M
        carry = s // M
    res[0] = carry

    res = res[-N:]
    return res


def sub(first_value, second_value, M, N):
    a = first_value[-N:]
    b = second_value[-N:]
    max_len = max(len(a), len(b))
    a = [0] * (max_len - len(a)) + a
    b = [0] * (max_len - len(b)) + b

    res = [0] * max_len
    borrow = 0
    for i in range(max_len - 1, -1, -1):
        val = a[i] - b[i] - borrow
        if val < 0:
            val += M
            borrow = 1
        else:
            borrow = 0
        res[i] = val

    if borrow == 1:
        add_back = [M - 1] * max_len
        carry = 1
        for i in range(max_len - 1, -1, -1):
            val = res[i] + add_back[i] + carry
            res[i] = val % M
            carry = val // M

    res = res[-N:]
    return res


def mul_small(a, k, M, N):
    if k == 0 or a == [0]:
        return [0] * N

    a = a[-N:]
    res = [0] * (len(a) + 1)
    carry = 0
    for i in range(len(a) - 1, -1, -1):
        prod = a[i] * k + carry
        res[i + 1] = prod % M
        carry = prod // M
    res[0] = carry

    res = res[-N:]
    return res


def times(first_value, second_value, M, N):
    a = first_value[-N:]
    b = second_value[-N:]
    if a == [0] * len(a) or b == [0] * len(b):
        return [0] * N

    res = [0] * (2 * N)
    len_a = len(a)
    len_b = len(b)

    for i in range(len_a - 1, -1, -1):
        carry = 0
        for j in range(len_b - 1, -1, -1):
            idx = i + j + 1
            total = res[idx] + a[i] * b[j] + carry
            res[idx] = total % M
            carry = total // M
        res[i] += carry

    res = res[-N:]
    return res


def div(first_value, second_value, M, N):
    if compare(second_value, [0]) == 0:
        return [0] # division by zero

    a = first_value[-N:]
    b = second_value[-N:]
    a = normalize(a)
    b = normalize(b)
    if compare(a, b) < 0:
        return [0] * N

    quotient = []
    remainder = [0]
    for digit in a:
        
        remainder = remainder + [digit]
        remainder = normalize(remainder)

        lo, hi = 0, M - 1
        q = 0
        while lo <= hi:
            mid = (lo + hi) // 2
            prod = mul_small(b, mid, M, N)
            if compare(prod, remainder) <= 0:
                q = mid
                lo = mid + 1
            else:
                hi = mid - 1
        quotient.append(q)
        if q:
            remainder = sub(remainder, mul_small(b, q, M, N), M, N)

    quotient = quotient[-N:]
    return quotient


a = from_int(2441, M, N)  
b = from_int(0, M, N)   

print(f"a = {a} --- {to_int(a, M)}")
print(f"b = {b} --- {to_int(b, M)}")

print("Sum:", sum(a, b, M, N), "---", to_int(sum(a, b, M, N), M))
print("Sub:", sub(a, b, M, N), "---", to_int(sub(a, b, M, N), M))
print("Mul:", times(a, b, M, N), "---", to_int(times(a, b, M, N), M))
print("Div:", div(a, b, M, N), "---", to_int(div(a, b, M, N), M))

    \end{lstlisting}

\end{document}