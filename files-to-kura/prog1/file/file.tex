\documentclass{article}[14pt]
\usepackage[T2A]{fontenc}
\usepackage{geometry}
\usepackage{fancyhdr}
\usepackage{tocloft}
\usepackage{hyperref}
\usepackage{titlesec}
\usepackage{graphicx}
\usepackage{caption}
\usepackage{wrapfig}
\usepackage[english, russian]{babel}
\geometry{
	a4paper,
	top=20mm, 
	right=25mm, 
	bottom=25mm, 
	left=25mm
}
\usepackage{amssymb}
\usepackage{fancyvrb}
\usepackage{fvextra}
\usepackage{ dsfont }
\usepackage{amsmath}
\usepackage{listings}
\usepackage{xcolor}
\usepackage{listingsutf8}
\usepackage{pdfpages}
\usepackage{lipsum}
\usepackage{makecell}
\usepackage{float}
\usepackage{longtable}


\lstset{ 
	basicstyle=\ttfamily\small,
	keywordstyle=\color{blue},
	inputencoding=utf8,              
	extendedchars=true,
	stringstyle=\color{red}
	commentstyle=\color{green},
	showspaces=false,         
	showstringspaces=false,  
	numbers=left,              
	numberstyle=\tiny\color{gray},
	breaklines=true,          
	frame=single,           
	captionpos=b,             
	tabsize=4,
	escapeinside={(*@}{@*)}, % для работы с нестандартными текстами
	literate={ё}{{\"o}}1 {Ё}{{\"O}}1 {«}{{\guillemotleft}}1 {»}{{\guillemotright}}1
	{А}{{\CYRA}}1 {Б}{{\CYRB}}1 {В}{{\CYRV}}1 {Г}{{\CYRG}}1 {Д}{{\CYRD}}1 
	{Е}{{\CYRE}}1 {Ж}{{\CYRZH}}1 {З}{{\CYRZ}}1 {И}{{\CYRI}}1 {Й}{{\CYRISHRT}}1 
	{К}{{\CYRK}}1 {Л}{{\CYRL}}1 {М}{{\CYRM}}1 {Н}{{\CYRN}}1 {О}{{\CYRO}}1 
	{П}{{\CYRP}}1 {Р}{{\CYRR}}1 {С}{{\CYRS}}1 {Т}{{\CYRT}}1 {У}{{\CYRU}}1 
	{Ф}{{\CYRF}}1 {Х}{{\CYRH}}1 {Ц}{{\CYRC}}1 {Ч}{{\CYRCH}}1 {Ш}{{\CYRSH}}1 
	{Щ}{{\CYRSHCH}}1 {Ъ}{{\CYRHRDSN}}1 {Ы}{{\CYRERY}}1 {Ь}{{\CYRSFTSN}}1 
	{Э}{{\CYREREV}}1 {Ю}{{\CYRYU}}1 {Я}{{\CYRYA}}1 
	{а}{{\cyra}}1 {б}{{\cyrb}}1 {в}{{\cyrv}}1 {г}{{\cyrg}}1 {д}{{\cyrd}}1 
	{е}{{\cyre}}1 {ж}{{\cyrzh}}1 {з}{{\cyrz}}1 {и}{{\cyri}}1 {й}{{\cyrishrt}}1 
	{к}{{\cyrk}}1 {л}{{\cyrl}}1 {м}{{\cyrm}}1 {н}{{\cyrn}}1 {о}{{\cyro}}1 
	{п}{{\cyrp}}1 {р}{{\cyrr}}1 {с}{{\cyrs}}1 {т}{{\cyrt}}1 {у}{{\cyru}}1 
	{ф}{{\cyrf}}1 {х}{{\cyrh}}1 {ц}{{\cyrc}}1 {ч}{{\cyrch}}1 {ш}{{\cyrsh}}1 
	{щ}{{\cyrshch}}1 {ъ}{{\cyrhrdsn}}1 {ы}{{\cyrery}}1 {ь}{{\cyrsftsn}}1 
	{э}{{\cyrerev}}1 {ю}{{\cyryu}}1 {я}{{\cyrya}}1               
}
\begin{document}
	
	\begin{flushright}
		{\Large{Шишмаков Владимир Олегович, гр. №5130201/30101}}
	\end{flushright}

    \begin{center}
        {\LARGE{\textbf{Программный модуль 1}}}
    \end{center}

    \large

    \section{Описание}

    \noindent \textbf{Название программы:} Калькулятор целых чисел из 10-ой в n-чную систему счисления.

    \noindent \textbf{Дано:} Два целых числа --- $x$ и $n$. $x$ --- число в десятичной системе счисления;
    $n$ --- система счисления, в которую требуется перевести x.

    \noindent \textbf{Требуется:} Перевести число $x$ в систему счисления $n$.

    \noindent \textbf{Выходные данные:} Вывести результат на экран в строковом формате.

    \bigskip

    \noindent \textbf{Ограничения:}

    \begin{itemize}
        \item $0 < n \leq 10$
        \item $x \geq 0$
    \end{itemize}

    \noindent \textbf{Спецификация:}

    {\Large{
    \begin{longtable}{|c|c|c|c|}
    \hline
    \textbf{№} & \textbf{Вход} & \textbf{Выход} & \textbf{Реакция программы} \\ \hline
    1 & 11 и 2 & 1101 & Вывод результата вычисления. Остановка. \\ \hline
    2 & 5 и 3 & 12 &  Вывод результата вычисления. Остановка. \\ \hline
    3 & 4 и 2 & 100 & Вывод результата вычисления. Остановка. \\ \hline
    4 & 16 и 5 & 31 & Вывод результата вычисления. Остановка. \\ \hline
    5 & 17 и 8 & 21 & Вывод результата вычисления. Остановка. \\ \hline
    6 & 0 и 2 & 0 & Вывод результата вычисления. Остановка. \\ \hline
    \end{longtable}}}

    \pagebreak

    \section{Блок-схема}

    \includegraphics[width=0.6\textwidth]{scheme.png}

    \pagebreak

    \section{Исходный код}

    \begin{lstlisting}
    def f(x, n):
    res = ''
    if x == 0:
        return '0'
    while x > 0:
        res = str(x % n) + res
        x //= n
    return res
    \end{lstlisting}
    
\end{document}