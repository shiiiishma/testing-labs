\documentclass{article}[14pt]
\usepackage[T2A]{fontenc}
\usepackage{geometry}
\usepackage{fancyhdr}
\usepackage{tocloft}
\usepackage{hyperref}
\usepackage{titlesec}
\usepackage{graphicx}
\usepackage{caption}
\usepackage{wrapfig}
\usepackage[english, russian]{babel}
\geometry{
	a4paper,
	top=20mm, 
	right=25mm, 
	bottom=25mm, 
	left=25mm
}
\usepackage{amssymb}
\usepackage{fancyvrb}
\usepackage{fvextra}
\usepackage{ dsfont }
\usepackage{amsmath}
\usepackage{listings}
\usepackage{xcolor}
\usepackage{listingsutf8}
\usepackage{pdfpages}
\usepackage{lipsum}
\usepackage{makecell}
\usepackage{float}
\usepackage{longtable}


\lstset{ 
	basicstyle=\ttfamily\small,
	keywordstyle=\color{blue},
	inputencoding=utf8,              
	extendedchars=true,
	stringstyle=\color{red}
	commentstyle=\color{green},
	showspaces=false,         
	showstringspaces=false,  
	numbers=left,              
	numberstyle=\tiny\color{gray},
	breaklines=true,          
	frame=single,           
	captionpos=b,             
	tabsize=4,
	escapeinside={(*@}{@*)}, % для работы с нестандартными текстами
	literate={ё}{{\"o}}1 {Ё}{{\"O}}1 {«}{{\guillemotleft}}1 {»}{{\guillemotright}}1
	{А}{{\CYRA}}1 {Б}{{\CYRB}}1 {В}{{\CYRV}}1 {Г}{{\CYRG}}1 {Д}{{\CYRD}}1 
	{Е}{{\CYRE}}1 {Ж}{{\CYRZH}}1 {З}{{\CYRZ}}1 {И}{{\CYRI}}1 {Й}{{\CYRISHRT}}1 
	{К}{{\CYRK}}1 {Л}{{\CYRL}}1 {М}{{\CYRM}}1 {Н}{{\CYRN}}1 {О}{{\CYRO}}1 
	{П}{{\CYRP}}1 {Р}{{\CYRR}}1 {С}{{\CYRS}}1 {Т}{{\CYRT}}1 {У}{{\CYRU}}1 
	{Ф}{{\CYRF}}1 {Х}{{\CYRH}}1 {Ц}{{\CYRC}}1 {Ч}{{\CYRCH}}1 {Ш}{{\CYRSH}}1 
	{Щ}{{\CYRSHCH}}1 {Ъ}{{\CYRHRDSN}}1 {Ы}{{\CYRERY}}1 {Ь}{{\CYRSFTSN}}1 
	{Э}{{\CYREREV}}1 {Ю}{{\CYRYU}}1 {Я}{{\CYRYA}}1 
	{а}{{\cyra}}1 {б}{{\cyrb}}1 {в}{{\cyrv}}1 {г}{{\cyrg}}1 {д}{{\cyrd}}1 
	{е}{{\cyre}}1 {ж}{{\cyrzh}}1 {з}{{\cyrz}}1 {и}{{\cyri}}1 {й}{{\cyrishrt}}1 
	{к}{{\cyrk}}1 {л}{{\cyrl}}1 {м}{{\cyrm}}1 {н}{{\cyrn}}1 {о}{{\cyro}}1 
	{п}{{\cyrp}}1 {р}{{\cyrr}}1 {с}{{\cyrs}}1 {т}{{\cyrt}}1 {у}{{\cyru}}1 
	{ф}{{\cyrf}}1 {х}{{\cyrh}}1 {ц}{{\cyrc}}1 {ч}{{\cyrch}}1 {ш}{{\cyrsh}}1 
	{щ}{{\cyrshch}}1 {ъ}{{\cyrhrdsn}}1 {ы}{{\cyrery}}1 {ь}{{\cyrsftsn}}1 
	{э}{{\cyrerev}}1 {ю}{{\cyryu}}1 {я}{{\cyrya}}1               
}
\begin{document}
	
	\begin{flushright}
		{\Large{Шишмаков Владимир Олегович, гр. №5130201/30101}}
	\end{flushright}

    \begin{center}
        {\LARGE{\textbf{Программный модуль 2}}}
    \end{center}

    \large

    \section{Описание}

    \noindent \textbf{Название программы:} Генерация случайных квадратных матриц одинаковой размерности 
    и их перемножение.

    \noindent \textbf{Дано:} Целое положительное число $n$ --- размер квадратных матриц.

    \noindent \textbf{Требуется:} Сгененрировать 2 случайных квадратных матрицы размерности $n \times n$ и перемножить их.

    \noindent \textbf{Выходные данные:} Выведенная матрица-результат на экран.

    \noindent \textbf{Ограничения:} $n > 0$.

    \noindent \textbf{Спецификация:}

    {\large{
    \begin{longtable}{|p{0.3cm}|p{1.1cm}|p{8cm}|p{6cm}|}
    \hline
    \textbf{№} & \textbf{Вход} & \textbf{Выход} & \textbf{Реакция программы} \\ \hline
    1 & $n > 0$ &  Вывод матрицы C & Программа генерирует две матрицы A и B, соответственно, после чего последовательно 
    выводит их. Вычисляется результирующая матрица C путем умножения матриц A и B и выводится на экран. \\ \hline
    2 & $n = 0$ & Вывод исключения вида java.lang.ArrayIndexOutOfBoundsException & Программа обнаружает ошибку
    и <<выбрасывает>> исключение. \\ \hline
    3 & $n < 0$ & Вывод исключения вида java.lang.NegativeArraySizeException & Программа обнаружает ошибку и 
    <<выбрасывает>> исключение. \\ \hline
    \end{longtable}}}

    \pagebreak

    \section{Блок-схема}

    \includegraphics[width=0.8\textwidth]{scheme.png}
    

    \pagebreak

    \section{Исходный код}

    \begin{lstlisting}
import java.util.Random;

public class MatrixMult {

    // Метод генерации случайной матрицы
    public static double[][] generateMatrix(int rows, int cols) {
        double[][] matrix = new double[rows][cols];
        Random rand = new Random();
        for (int i = 0; i < rows; i++) {
            for (int j = 0; j < cols; j++) {
                matrix[i][j] = rand.nextDouble();
            }
        }
        return matrix;
    }

    // Проверка, что матрица не null
    public static boolean isNotNull(double[][] matrix) {
        return matrix != null;
    }

    // Простой метод умножения матриц
    public static double[][] multiply(double[][] firstMatrix, double[][] secondMatrix) {
        int n1 = firstMatrix.length;
        int m1 = firstMatrix[0].length;
        int n2 = secondMatrix.length;
        int m2 = secondMatrix[0].length;

        if (m1 != n2) {
            throw new IllegalArgumentException("Размерности матриц не совпадают для умножения");
        }

        double[][] result = new double[n1][m2];

        for (int i = 0; i < n1; i++) {
            for (int j = 0; j < m2; j++) {
                double sum = 0;
                for (int k = 0; k < m1; k++) {
                    sum += firstMatrix[i][k] * secondMatrix[k][j];
                }
                result[i][j] = sum;
            }
        }
        return result;
    }

    public static void printMatrix(double[][] A) {
        if (isNotNull(A)) {
            for (int i = 0; i < A.length; i++) {
                for (int j = 0; j < A[0].length; j++) {
                    System.out.print(A[i][j] + "\t");
                }
                System.out.println();
            }
        }
        else {
            System.out.println("It is null matrix!");
        }
        System.out.println();
    }

    public static void main(String[] args) {
        int n = 5;

        double[][] A = generateMatrix(n, n);
        double[][] B = generateMatrix(n, n);
        
        System.out.println("First matrix:");
        printMatrix(A);

        System.out.println("Second matrix:");
        printMatrix(B);

        double[][] C = multiply(A, B);

        System.out.println("Multiply matrix:");
        printMatrix(C);

    }

}

    \end{lstlisting}

\end{document}